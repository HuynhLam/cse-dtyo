%
% Template for Department of Electrical and Information Engineering Diploma Thesis v1.1.2013
% Authors: Mika Korhonen (original author), Pekka Pietikäinen, Christian Wieser, Teemu Tokola and Juha Kylmänen.
% If you make any improvements to this template, please contact ouspg@ee.oulu.fi
%

\documentclass[a4paper, 12pt,titlepage]{dithesis}
\usepackage[english,finnish]{babel}
\usepackage[utf8]{inputenc}
\usepackage[T1]{fontenc}  
\usepackage{times}
\usepackage{tabularx}
\usepackage{graphicx}
\usepackage{float}
\usepackage{enumerate}
\usepackage{placeins}
\usepackage{fancybox}
\usepackage{verbatim}
\usepackage{longtable}
\usepackage{di}
\usepackage[hyphens]{url}
\usepackage{boxedminipage}
\usepackage{subfigure}
\usepackage{multirow}
\usepackage{amsfonts}
\usepackage{xcolor}
\tolerance=500

%\usepackage[a4paper,margin=2.5cm,dvips]{geometry}
%\geometry{papersize={210mm,297mm}}
%dvipdf -sPAPERSIZE=a4

% The following code removes %-signs with URL:s longer than 72 chars
\begingroup
\makeatletter
\g@addto@macro{\UrlSpecials}{%
  \endlinechar=13 \catcode\endlinechar=12
  \do\%{\Url@percent}\do\^^M{\break}}
 \catcode13=12 %
 \gdef\Url@percent{\@ifnextchar^^M{\@gobble}{\mathbin{\mathchar`\%}}}%
\endgroup %

%\selectlanguage{finnish}

\otsikko{Suomenkielinen nimi}
\title{English title}

\etunimi{Teemu}
\sukunimi{Teekkari}
\valvoja{prof. Juha Röning}
\koulutusohjelma{information} % {information | electrical}
\vuosi{2016}
\tyo{Master} % {Bachelor | Master}
\kieli{english} % {finnish | english}

\begin{document}

\begin{titlepage}
	\centering{\includegraphics*[width=0.3\textwidth]{uni_logo}\\}
	{\sffamily\fontsize{9}{1pt}\selectfont FACULTY OF INFORMATION TECHNOLOGY AND ELECTRICAL ENGINEERING\\}
	%{\sffamily\fontsize{9}{1pt}\selectfont TIETO- JA SÄHKÖTEKNIIKAN TIEDEKUNTA\\}
	\vspace{65 mm}
	{\textbf{\fontsize{16}{19pt}\selectfont \getfirstname\ \getlastname }\\}
	\vspace{15 mm}
	{\textbf{\fontsize{18}{22pt}\selectfont FIRST FRONT PAGE TITLE LINE\\ADDITIONAL IF NEEDED\\}}
	\vspace{60 mm}
	{\fontsize{14}{17}\selectfont Master's Thesis \\Degree Programme in Computer Science and Engineering \\ Month 2016\\}
	%{\fontsize{14}{17}\selectfont Diplomityö \\Tietotekniikan tutkinto-ohjelma \\ Month 2016\\}
\end{titlepage}

\selectlanguage{english}

\begin{abstract}
This is a sample abstract.

\keywords sample, keywords

\end{abstract}

\selectlanguage{finnish}
\begin{tiivistelma}
Esimerkkitiivistelmä

\avainsanat esimerkki, sanoja
\end{tiivistelma}

\selectlanguage{english}
%\selectlanguage{finnish}

\sisluettelo
%\tableofcontents

\otsake{FOREWORD}
This \LaTeX -template has been used by various people at department
since the late 1990's, and has slowly improved over time.  It is still
somewhat rough at the edges, but hopefully will be helpful in reducing
some of the pain involved in writing a diploma thesis.

Contributors to the template include Mika Korhonen (original author),
Pekka Pietikäinen, Christian Wieser and Teemu Tokola.  If you make any
improvements to this template, please contact ouspg@ee.oulu.fi, and we
will try to include them in further revisions.

The template was updated during the summer of 2013 by Juha Kylmänen.
%\allekirjoitus{Oulu, Finland \today}

\otsake{ABBREVIATIONS}

\setlongtables
\begin{longtable}[l]{p{3cm}p{0.7\textwidth}}

% Add your abbreviations to abbreviations.tex
LAN & local area network\\
IP & Internet protocol\\
GUI & graphical user interface\\
HTTP & hypertext transfer protocol\\
UDP & user datagram protocol\\
CPU & central processing unit \\
RAM & random-access memory\\
FTP & file transfer protocol\\
WWW & world wide web \\
API & application programming interface \\
NAT & network address translation \\
CLI & command line interface \\
DPI & dots per inch \\


\end{longtable}
\setcounter{table}{0}

%Johdanto
\chapter{Introduction}
\sivunumerot
Lorem ipsum dolor sit amet, consectetur adipiscing elit. Pellentesque lobortis eget dolor nec vehicula.
Donec pretium dui at bibendum accumsan \cite{cheswick}. Aliquam sollicitudin pharetra felis, in euismod velit volutpat 
vitae. Praesent iaculis id est sed ultrices. Nunc pretium euismod dolor, nec molestie elit ultricies a. 
Praesent laoreet mi metus, eu luctus dui lobortis vel. Vivamus sit amet elementum augue, ac lacinia leo.

Aenean vel est pellentesque, malesuada dolor nec, porta diam. Quisque molestie quis nibh quis facilisis. 
Nam placerat, ante imperdiet adipiscing rhoncus, quam augue consequat ligula, eu porttitor velit turpis 
a sem. Donec nibh dolor, sagittis in metus at, porta lacinia risus. Vestibulum quis rhoncus massa, eu 
eleifend metus. Curabitur vehicula malesuada enim at rutrum. Praesent eu velit porta, lacinia tortor ac, 
fermentum libero. \cite{kamara, al-shaer}


\section{Sample section with a table reference}

Ut hendrerit volutpat felis vitae aliquam. Duis quis augue urna. In sollicitudin lacinia elit, 
non ultrices dui tristique eu. In hac habitasse platea dictumst. Nullam mi sapien, sagittis non 
mi in, gravida lobortis ante. A sample latex table can be seen in Table~\ref{tab:sample_table}.


\begin{table}[!ht]
% Add some padding to the table cells:
\def\arraystretch{1.1}%
\begin{center}
  \caption{Sample table}
  \label{tab:sample_table}
  \begin{tabular}{| l | c | }
    \hline
    Sample & table \\
    \hline
    Sample & table \\
    Sample & table \\
    \hline
  \end{tabular}

  \end{center}
\end{table}

\section{Changelog}

\begin{itemize}
\item Changed \textit{\textbackslash{chapter's}} \textbackslash{newpage} to 
\textbackslash{clearpage} to prevent floats from wandering to the beginning of the next chapter

\item Added \textbf{[hyphens]} to the url package to prevent margin overflow with 
long urls

\item Added \textbf{multirow} package to make multirow and multicolumns possible

\item Added some helpful source code comments

\item Makefile for pdflatex and bibtex to automate pdf compilation

\item Abbreviations are autosorted by the Makefile

\item Added a bit of extra padding to the sample table
\end{itemize}

\subsection{Sample subsection with a figure reference}

Sed erat neque, cursus ac feugiat ac, sollicitudin
ut odio. Maecenas vel turpis rhoncus, euismod nisl ac, tincidunt ipsum. Curabitur fermentum vel
turpis ac lobortis. Cras a justo vitae diam volutpat blandit. Maecenas faucibus nibh a neque 
semper ullamcorper. Suspendisse in est vulputate, fermentum odio nec, pharetra augue. Fusce at
consequat arcu, sed hendrerit enim. Pellentesque id suscipit nibh, id pretium erat. 

Nam eget libero neque. Nullam commodo cursus turpis mollis cursus. Curabitur est tellus,
pellentesque eu velit sed, ullamcorper gravida felis. Proin vel cursus risus, at scelerisque 
justo. Quisque rutrum justo at ultricies auctor. A sample latex figure can be seen in
Figure~\ref{fig:oylogoe}. If your pictures appear grainy, you probably have too low dots
per inch (DPI) value.

\footnotetext[1]{Sample footnote}

%Pictures in .eps if you use latex, .pdf or .png if you use pdflatex. Don't specify the extension so you can use both!
\begin{figure}[ht]
  \begin{center}
    \includegraphics*[width=0.3\textwidth]{oylogoe}
  \end{center}
  \caption{A picture}
  \label{fig:oylogoe}
\end{figure}

 % ./introduction.tex

\chapter{Implementing something}
% Implementation Chapter

Your implementation.
  % ./implementation.tex

\chapter{Testing something}
% Testing Chapter

Your testing.
  % ./testing.tex

\chapter{Discussion}
% Discussion Chapter

Your discussion.
  % ./discussion.tex

\chapter{Conclusion}
% Conclusions Chapter

The best thesis evar.
  % ./conclusion.tex

\bibliographystyle{di}
\bibliography{di}
\end{document}
